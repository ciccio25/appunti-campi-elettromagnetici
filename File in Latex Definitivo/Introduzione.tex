\section*{Introduzione}

 

Appunti ordinati, con approfondimenti passo-passo, del corso di Campi Elettromagnetici per il corso di laurea in Ingegneria Elettronica 
presso l’Università Politecnica delle Marche. \newline
        



Le fonti degli appunti sono le seguenti: 

\begin{itemize}
    
    \item Fields and Waves Electronics, 3rd Edition, Ramo, 1994 
    \\(negli appunti sarà abbreviato come FWE) 
    
    \item  Fundamentals of Applied Electromagnetics (8th edition), Fawaz T. Ulaby 
    \\(negli appunti sarà abbreviato come FAE) 

    \item Fondamenti di campi elettromagnetici - Teoria ed applicazioni, Fawwaz T. Ulaby, McGraw-Hill, 2006 
    \\(negli appunti sarà abbreviato come FCE: è la traduzione in italiano di FAE)
    
    \item Slide del corso del prof Antonio Morini, Campi elettromagnetici A.A. 2022/2023, PPT aggiornati al 2023
    \\(negli appunti sarà indicato come PPT) 
    

\end{itemize}

\begin{tcolorbox}
    Negli appunti ci saranno delle piccole appendici dentro a questi mini-paragrafi  
    su come si leggono in italiano le varie formule matematiche, 
    a prova di imbecille,  
    e lascerò link su possibili approfondimenti matematici usando le animazioni, 
    così è più facile comprendere la materia.
\end{tcolorbox}


Per le lettere greche che ci saranno nel corso, vi consiglio di visitare questo sito \\ 
\url{https://www.rapidtables.org/it/math/symbols/greek_alphabet.html} \break 
in cui è disponibile anche la pronuncia vocale delle lettere. \newline

È consigliato studiare e superare prima l’esame di analisi matematica 2, quindi anche analisi matematica 1,  e fondamenti di elettromagnetismo, ma se stai leggendo questi appunti, molto probabilmente hai saltato almeno uno tra questi. \newline 

Non ti preoccupare, ci sono io qui che cerco a darti una mano. \newline 

O maschio o femmina, bro sono qui con te (virtualmente) a superare questo esame. \newline 

Per qualsiasi domanda, scrivimi a \href{mailto:rossini.stefano.appunti@gmail.com}{rossini.stefano.appunti@gmail.com} \newline

Buono studio e buona lettura \newline

\newpage 





